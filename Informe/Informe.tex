\documentclass[12pt, a4paper]{article}

\usepackage{listings}

\usepackage{xcolor}
\usepackage{multicol}

\usepackage{graphicx}

\usepackage{fancyhdr}

\usepackage{geometry}

\usepackage{verbatim}

\geometry{
    left=2cm,
    right=2cm,
    top=2.5cm,
    bottom=3.5cm
}

\pagestyle{fancy}
\fancyhf{}
\fancyfoot[R]{\thepage}
\fancyhead[L]{\bf\large{Moogle!}\color{blue}\hrule}
\setlength{\headheight}{40pt}
\renewcommand{\headrulewidth}{0pt}

\definecolor{codegreen}{rgb}{0,0.6,0}
\definecolor{codegray}{rgb}{0.5,0.5,0.5}
\definecolor{codepurple}{rgb}{0.58,0,0.82}
\definecolor{backcolour}{rgb}{0.95,0.95,0.92}

\lstset{
	language=[Sharp]C,
    backgroundcolor=\color{backcolour},   
    commentstyle=\color{codegreen},
    keywordstyle=\color{purple},
    numberstyle=\tiny\color{codegray},
    stringstyle=\color{codepurple},
    basicstyle=\ttfamily\footnotesize,
    breakatwhitespace=false,         
    breaklines=true,                 
    captionpos=b,                    
    keepspaces=true,                 
    numbers=left,                    
    numbersep=5pt,                  
    showspaces=false,                
    showstringspaces=false,
    showtabs=false,                  
    tabsize=4
}

\renewcommand{\emph}{\textcolor{blue}}

\begin{document}

\begin{titlepage}
    \centering
    
    \begin{figure}[h]
        \center
        \includegraphics[width=5cm]{matcom.jpg}
        
        \Large Facultad de Matématica y Computación

        \large Universidad de la Habana
    \end{figure}

    
    \vspace{2cm}
    

    \vspace{2cm}

    {\Huge\bf Proyecto de Programación I \par Moogle!}

    
    \vfill

    \large Ernesto Abreu Peraza
    
    \small Grupo: C-121
    
    \small Curso: 2023
\end{titlepage}

\newpage

\renewcommand{\abstractname}{Descripción}

\begin{abstract}
    Moogle! es una aplicación \textit{"totalmente original"} cuyo propósito es buscar inteligentemente un texto en un
conjunto de documentos.

Es una aplicación web, desarrollada con tecnología .NET Core 6.0, específicamente usando Blazor como
framework web para la interfaz gráfica, y en el lenguaje C\#. La aplicación está dividida en dos componentes
fundamentales:
\begin{itemize}
    \item \emph{MoogleServer} es un servidor web que renderiza la interfaz gráfica y sirve los resultados.
    \item \emph{MoogleEngine} es una biblioteca de clases donde está... ehem... casi implementada la lógica del algoritmo de búsqueda.
\end{itemize}
\end{abstract}

\section*{Correr y usar el proyecto}

Para correr el proyecto debes usar el comando \emph{dotnet watch run --project MoogleServer} en Windows
y \emph{make dev} en Linux. En la carpeta \emph{Content} deberán aparecer los documentos (en formato *.txt) en los que el
usuario va a realizar la búsqueda. En la casilla donde aparece \textit{Introduzca la búsqueda} el usuario va a escribir
que desea buscar y basta con apretar el botón \textit{Buscar} para que Moogle! haga su trabajo.


\section*{Arquitectura del proyecto}

\subsection*{Características de clases}

\begin{itemize}
    \item \emph{Moogle}: procesar consulta.
    \item \emph{TF\_IDF}: calcular tf-idf para los documentos y para la query, calcular coseno del angulo entre dos vectores.
    \item \emph{DocumentReader}: leer, normalizar y obtener de los documentos, título, texto y palabras.
    \item \emph{StringUtil}: procesar, calcular, modificar y obtener informacion a partir de texto util para alguna de las funcionalidades de la aplicación.
    \item \emph{Matrix}: Definir y multiplicar matrices
    \item \emph{Vector}: Definir, calcular modulo y multiplicar vectores
\end{itemize}

\section*{Precálculo}

Term frequency-Inverse document frequency (TF\_IDF) es una medida numérica que expresa cuan
relevante es una palabra para un documento en una colección de documentos.

\begin{eqnarray*}
    TF\_IDF[t,d] = TF[t,d] * IDF[t] \\
    TF[t,d] = \frac{f[t]}{maxFrequency} \\
    IDF[t] = \log_{10}{\frac{numberOfDocuments}{Df[t]}}
\end{eqnarray*}

Siendo $t$ una palabra y $d$ un documento. $f[t]$ la frecuencia de la palabra $t$ en el documento $d$ 
y $maxFrequency$ es el máximo de los $f[t]$. $numberOfDocuments$ el total de documentos y 
$Df[t]$ es la cantidad de documentos en los que aparece la palabra $t$.

\begin{itemize}
    \item Cuando el programa empieza a correr se ejecuta el método \texttt{TF\_IDF.Compute()} del archivo
    \texttt{TF\_IDF.cs}. Este método lee el texto de cada documento que aparece en la carpeta \emph{Content} y de el
    extrae todas las palabras. Se calcula el \texttt{TF\_IDF} para cada palabra en cada documento

\begin{lstlisting}
public static void Compute()
{
    /* Precalcula el TF_IDF */
    
    /* Arreglo con los nombres de los documentos */
    documentsName = DocumentReader.DocumentsNameList("..\\Content");
    
    int numberOfDocuments = documentsName.Length;
   
    /* Diccionario [nombre de documento => indice] */
    for (int i = 0; i < numberOfDocuments; i++)
    {
        Document[documentsName[i]] = i;
    }
    
    /* Arreglo con las palabras del los documentos */
    words = DocumentReader.WordsList(documentsName);
    
    int numberOfWords = words.Length;
    
    /* Diccionario [palabra => indice] */
    for (int i = 0; i < numberOfWords; i++)
    {
        Word[words[i]] = i;
    }
    
    Matrix TF = ComputeTF();
    tf = TF;
    
    Vector IDF = ComputeIDF();
    idf = IDF;
    
    Matrix TF_IDF = new Matrix(numberOfWords, numberOfDocuments);
    
    /* Multiplicar TF por IDF */
    for (int j = 0; j < numberOfDocuments; j++)
        for (int i = 0; i < numberOfWords; i++)
            TF_IDF[i, j] = TF[i, j] * IDF[i];
    
    tf_idf = TF_IDF;
}
\end{lstlisting}

\item Los métodos \texttt{ComputeTF()} y \texttt{ComputeIDF()} calculan el TF y el IDF respectivamente.
\end{itemize}

\subsection*{Consulta}

\begin{itemize}
    \item Luego que la aplicación inicie, cuando se realize una consulta ( búsqueda ), el proyecto llama al método
    \texttt{Moogle.Query()} del archivo \texttt{Moogle.cs} donde se calcula el TF\_IDF para la consulta a traves del
    método \texttt{TF\_IDF.ComputeQueryTF\_IDF()} y se compara con el TF\_IDF de cada documento dandole
    una puntuación a cada uno y devolviendo una lista con los nombres de los documentos que más
    relevancia tienen. La puntuación sería el coseno del ángulo entre el vector formado con el TF\_IDF de la
    consulta y el TF\_IDF del documento. La similitud de los documentos es mayor mientras mas
    se acerca a 1 el coseno del angulo entre ellos, visto de otra forma mientras mas pequeño es el angulo
    entre ambos. Para esto se utilizan dos fórmulas de dot product. Para dos vectores a y b: $dotProduct
    = a_1 * b_1 + a_2 * b_2 + ... + a_n* b_n$ , $dotProduct = \cos(a,b) * |a|*|b|$.

\begin{lstlisting}
public static (float, int)[] VectorialModel(Vector queryTF_IDF)
{
    /* Devuelve un arreglo que contiene para cada documento el coseno del angulo
    entre el vector de la query y el del documento */
    (float, int)[] vectorialModel = new (float, int)[tf_idf.columns];
    for (int j = 0; j < tf_idf.columns; j++)
    {
        Vector v = new Vector(tf_idf.rows);
        for (int i = 0; i < tf_idf.rows; i++)
        {
            v[i] = tf_idf[i, j];
        }
        float score = 0;
        if (queryTF_IDF.Module() * v.Module() != 0)
        score = Vector.Dot_Product(queryTF_IDF, v) / (queryTF_IDF.Module() *
        v.Module());
        vectorialModel[j] = (score, j);
    }
    return vectorialModel;
}
...
...
...
/* Definicion de producto escalar entre dos vectores */
static public float Dot_Product(Vector a, Vector b)
{
    if (a.Dimensions != b.Dimensions)
        /* Exception */
        return 0;
    
    float dot_product = 0;
    for (int i = 0; i < a.Dimensions; ++i)
        dot_product += a[i] * b[i];
    
    return dot_product;
}
...
...
...
/* Definicion de modulo de un vector */
public float Module()
{
    float module = 0;
    for (int i = 0; i < this.Dimensions; i++)
    {
        module += this[i] * this[i];
    }
    return (float)Math.Sqrt(module);
}
\end{lstlisting}

\item También se muestra un fragmento por cada documento donde se puede apreciar en este una relación
del documento con la consulta.

\end{itemize}

\subsection*{Funcionalidades extras}

Otra funcionalidad del Moogle! es que dará una sugerencia de búsqueda en caso de que el usuario quizás
cometió un error al escribir la consulta. Para esto usamos un algoritmo de Edit Distance conocido como
Levenshtein distance y a este costo lo dividimos por el Longest Common Prefix. El propósito de este último
proviene de la idea de que es más probable equivocarse en las últimas letras que en las primeras. Así distra
está más cerca de dijkstra que de citra.

\begin{lstlisting}
public static float Distance(string a, string b)
{
    /* Calcula la distancia entre dos cadenas de caracteres */
    return EditDistance(a, b) / LongestCommonPrefix(a, b);
}

public static float EditDistance(string a, string b)
{
    /* Calcula el EditDistance para dos cadenas de caracteres */
    int[,] dp = new int[a.Length + 1, b.Length + 1];
    for (int i = 0; i <= a.Length; i++)
        for (int j = 0; j <= b.Length; j++)
        {
            if (i == 0 || j == 0)
            {
                dp[i, j] = Math.Max(i, j);
            }
            else
            {
                dp[i, j] = Int32.MaxValue;
                if (a[i - 1] == b[j - 1])
                   dp[i, j] = dp[i - 1, j - 1];
                else
                {
                    dp[i, j] = Math.Min(dp[i, j], dp[i - 1, j] + 1);
                    dp[i, j] = Math.Min(dp[i, j], dp[i, j - 1] + 1);
                    dp[i, j] = Math.Min(dp[i, j], dp[i - 1, j - 1] + 1);
                }
            }
        }
    return (float)dp[a.Length, b.Length];
}

public static float LongestCommonPrefix(string a, string b)
{
    /* Calcula el LongestCommonPrefix para dos cadenas de caracteres */
    for (int i = 0; i < Math.Min(a.Length, b.Length); i++)
        if (a[i] != b[i])
            return (float)(i + 1);
    
    return (float)Math.Min(a.Length, b.Length) + 1;
}
\end{lstlisting}

\end{document}